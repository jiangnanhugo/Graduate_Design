\addcontentsline{toc}{section}{摘要}%在目录中添加“参考文献”的标记
\keywords{纳维-斯托克斯方程;计算机动画;火焰模拟}
{Navier-Stokes;\ Computer Animation;\ Fire Simulation}

\begin{abstract_ch}
  火焰动画的仿真和渲染作为流体计算力学在计算机图形学的一个重要应用之
  一,一直是一个很有挑战性的研究课题。近年来随着计算机硬件能力的提升
  和GPU并行计算的快速发展,具有高度真实感的实时火焰的模拟已经成为了研究
  的一个热点。

  本文首先介绍了火焰模拟的主要步骤,回顾了计算机图形学领域中对火焰的建模、绘制方法,以及
  火焰和环境交互下的主要模拟方法和算法。在建模方法中着重介绍了基于物理的火焰建模方法,即基于计算流体力学的模拟方法。然后重点讨论了火焰和环境交互时的相关研究成果。最后本文还展望了对于流体动画模
  拟未来发展的几个方向。
\end{abstract_ch}

\begin{abstract_en}
  As one of the most important applications of Computational Fliud
  Dynamics in computer graphics, simulation of fire animation has been
  a chanllenging research topic. With the improvment of the computer
  hardware and rapid development of GPU parallel computing, simulation
  of highly realastic flame has become a research hotspot.


 Firstly, the main steps of flame simulation were presented, then we
will review the modeling and drawing methods of flame in the field of
computer graphics as well as the main simulation methods and
algorithms under the interaction of fire and environment. In the
modeling approach, we will focus on physics-based fire modeling, which
is the simulation methods based on computational fluid dynamics.
Afterwards, we will put emphasis to the relevant research results
about flame and environment interaction. Finally, this paper looked
ahead to the future development of the fluid animation in several
directions.

\end{abstract_en}
\newpage
