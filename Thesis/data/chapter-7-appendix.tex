% 附页\emph{}
\chapter{攻读硕士学位期间取得的学术成果}
% 此处标题及内容请自行更改
\noindent 发表论文:

\noindent 1. \textbf{Nan Jiang}, Wenge Rong, Min Gao, Yikang Shen and Zhang Xiong. Exploration of Tree-based Hierarchical Softmax for Recurrent Language Models[C]. Proceedings of the Twenty-Sixth International Joint Conference on Artificial Intelligence (IJCAI), 2017, pp. 1951-1957. (已发表)

\noindent 2. \textbf{Nan Jiang}, Wenge Rong, Yifan Nie, Yikang Shen and Zhang Xiong. Event Trigger Identification with Noise Contrastive Estimation[J]. IEEE/ACM Transactions on Computational Biology and Bioinformatics, 2017, pp. 1-11.(已发表)

\noindent 3. \textbf{Nan Jiang}, Wenge Rong, Baolin Peng, Yifan Nie and Zhang Xiong. Modeling Joint Representation with Tri-Modal DBNs for Query and Question Matching[J]. IEICE Transactions on Information and Systems, 2016, 99(4): 927-935.(已发表)

\noindent 4. \textbf{Nan Jiang}, Wenge Rong, Baolin Peng, Yifan Nie and Zhang Xiong. An Empirical Analysis of Different Sparse Penalties
for Autoencoder in Unsupervised Feature Learning[C]. International Joint Conference on Neural Networks (IJCNN), 2015, pp. 1-8.(已发表)

\noindent 5. Zhen Xu, \textbf{Nan Jiang}, Bingquan Liu, Wenge Rong, Bowen Wu, Baoxun Wang, Xiaolong Wang and Zhuoran Wang. LSDSCC: a Large Scale Domain-Specific Conversational Corpus for Response Generation with Diversity Oriented Evaluation Metrics. The 2018 Conference of the North American Chapter of the Association for Computational Linguistics - Human Language Technologies (NAACL-HLT), 2018. (已录用)

\noindent 6. Yifan Qian, Wenge Rong, \textbf{Nan Jiang}, Jie Tang, Zhang Xiong. Citation regression analysis of computer science publications in different ranking categories and subfields. Scientometrics 110(3): 1351-1374 (2017)

\noindent 7. Yikang Shen, Wenge Rong, \textbf{Nan Jiang}, Baolin Peng, Jie Tang and Zhang Xiong. Word Embedding Based Correlation Model for Question/Answer Matching[C]. Proceedings of the Thirtieth {AAAI} Conference on Artificial Intelligence (AAAI), 2017, pp. 3511-3517.(已发表)

\noindent 8. \textbf{Nan Jiang}, Wenge Rong, Min Gao, Yikang Shen and Zhang Xiong. Exploration of Hierarchical Softmax for Recurrent Language Models[J]. ACM Transactions on Intelligent Systems and Technology (TIST).(在审)

\noindent 9. \textbf{Nan Jiang}, Wenge Rong, Libin Shi. Addressing Universal Replies in Response Generation Models with Marginal Regularization. 56th Annual Meeting of the Association for Computational Linguistics (ACL). (在审)

\chapter{致\quad 谢}
在2015年,我来到了北京航空航天大学计算机学院先进计算机技术教育工程研究中心,开始了我为期三年的研究生学习阶段。首先感谢荣文戈副教授当年的知遇之恩,没有导师三年以来的指点与意见,我也不会成为现在的我。

我能够在短短两年多的研究生阶段发表如此数量的科研论文,离不开荣老师的悉心指导和耐心改正我的错误。荣老师不仅为我设定了远大的目标,还真且关注我的水平的成长,每次都是设置一个能力可及的任务,不断锻炼我的写作技能和实验技能。最令人记忆深刻的是,在每次论文投稿前一周,每当深夜我将论文修订完一版本之后,荣老师总是立马修改,甚至在早上四点给我修改论文。顶着巨大的身体压力和时间压力,给我不断修改论文,还不断跟我捋顺论文思路,其耐心已经超越了我人生认识的所有老师。我的英语水平在本科二年级考完CET-6之后,没有逐渐生疏,反倒是经过导师不断的磨练得以日益提升。然而学术的生涯并不总是一帆风顺的,在很长一段时间的瓶颈期,荣老师在听完我不断的抱怨之后,仍然鼓励我,帮我树立做科研的自信心,使我明白不断加强自身知识和技术水平的重要性。经过整整大半年的低谷时期,总算迎来了一点新的成果,此时的我非常膨胀,自视甚高。此时荣老师又劝导我,过度自信和过度自我否定都是不可取的,人生路尚且长,我们需要走好每一步,不要因为别人的质疑而否定自己,更不能因为别人的赞美而吹嘘自己。自小离家求学,父母的管教未曾领受,导致现在的我存在诸多性格和学习态度上的缺点,荣老师在这两年里,如醍醐灌顶一般,将他的人生哲学授予我,实在是不可多得的财富。相比于那些物质上的利益,导师传授的这些生存哲学才是颠扑不破的道理。

还要感谢实验室的博士、师兄、师姐对我的呵护和指点。其中包括:陈虞君博士、张硕师姐、聂一凡大师兄、沈驿康师兄。还有已经毕业的宋欣、谢维柱师兄。令人记忆深刻的是我当初投稿阶段,实验需要很多台计算设备,他们一听到我的需求,立马把他们的高精度计算平台借给我使用。有这样一群和谐友善的长辈护在身边,实在是人生之大幸。除此之外,我还要感谢可爱的师弟师妹们。是立彬、田川师弟。还要感谢邱晨和高志峰提供源源不断有关实验和算法的建议。

最后感谢我的父母,由于家境不是很好,父母从我三岁开始去上海打工,历时20余载,一直坚守在自己的岗位上,为了我的以后的未来,积攒下一点积蓄。尽管他们远在他乡,对我的学业也是非常的关心,每周都得电话通讯,报告进期的教授的知识和学业成绩。没有一句累了不想干了,我因有他们一直陪在我身边感到幸运,因他们的不懈付出而感到学业有成的重要性。